\documentclass{book}
\usepackage[utf8]{inputenc}
\usepackage{indentfirst}
\usepackage[brazil]{babel}
\usepackage{pifont}
\usepackage{graphicx}

\begin{document}

\pagenumbering{arabic}
\tableofcontents	
\newpage

\backmatter
\setcounter{chapter}{2}
\chapter{2 \ \ Análise Exploratória de Dados}

\section{Medidas de Posição}

As medidas de posição, tembém chamada de medidas de tendência central, possuem três formas diferentes para três situações distintas:

\begin{dingautolist}{182}

    \item Média
    \item  Moda
    \item  Mediana
    
\end{dingautolist}

\subsection{Média}
Existem dois tipos de média:\footnote{somatório...}

\begin{dinglist}{43}
\item POPULACIONAL, representada pela letra grega $\mu$
\item AMOSTRAL, representada por $\bar{x}$

\end{dinglist}

\textbf{1- Média}: (Dados não agrupados) \\
\\
Sejam os elementos $x_1, x_2,...,x_n$ de uma amostra, portanto``$n$" valores da variável $X$. A média aritmética da variável aleatória X é definida por, 

$$\bar{x} = \frac{x_1 + x_2 + ... + x_n}{n} = \frac{\sum_{i=1}^n{x_i}}{n}$$

\textbf{Exemplo:} Suponha o conjunto de dados que representa o peso ao nascer de bezerros de raça Nelore: 51,40,46,48,54,56,44,43,55 e 57.Determinar a média aritmética simples deste conjunto de dados

$$\bar{x}= \frac{51+40+46+48+54+56+44+43+55+57}{10} = \frac{494}{10} = 49,4$$

\textbf{2-Média:} (Dados agrupados em uma distribuição de frequência por valores simples)\\
\\

Usa-se a média aritmética dos valores $x_1,x_2,...,x_n$ ponderados pelas respectivas frequências absolutas: $f_1,f_2,...,f_n.$ Assim 

$$\bar{x}= \frac{x_1 f_1 + x_2 f_2 + ... + x_n f_n}{n} = \frac{\sum_{1=n}^n{x_i f_i}}{n}$$ 


\textbf{Exemplo:}

%\caption{Tabela 6: Distribuição do número de alunos em 20 turmas da UFCG}

\begin{table}[h]
\centering
    \caption{Cronograma} \\

\begin{tabular}{|l||c|c|c|c|c|c|}
\hline 
Fase & Março & Abril & Maio & Junho & Jullho \\
\hline
1    & center &       &        &        &      \\
2    &        & center& center &        &       \\ 
3    &        &       &center  & center &      \\ 
4    &        &       &        & center & center \\
5    &        &       &       &         & center \\
\hline
\hline

\end{tabular} 

\end{table}

%\caption{Cronograma} \\

%\begin{tabular}{|l||c|c|c|c|c|c|}
%\hline 
%Fase & Março & Abril & Maio & Junho & Jullho \\
%\hline
%1    & center &       &        &        &      \\
%2    &        & center& center &        &       \\ 
%3    &        &       &center  & center &      \\ 
%4    &        &       &        & center & center \\
%5    &        &       &       &         & center \\

%\end{tabular}

\begin{table}[h]
    \centering
    \caption{Tabela 6: Distribuição do número de alunos em 20 turmas da UFCG}\\
    
    \begin{tabular}{c|c|c|c|c|c|c|c|c|c|c|c|c|c|c}
    \hline
    $i$         & 1   & 2   & 3  & 4   & 5  & 6   & 7   & 8   & 9  & 10  & 11   &   12   & 13  & Total \\ 
    \hline
    Dados$(x_i)$ & 41& 42  & 43 & 44  & 45  &46  & 50  & 51  & 52  &54  &57   &  58   &60  & $\sum_{i=1}^{13}{f_i}$ \\
    \hline
    $(f_i)$ & 3 & 2 & 1 & 1 & 1 & 2 & 2 & 1 & 1 & 1 & 1 & 2 & 2 & 20 \\ 
    \hline
    $x_i f_i$ & 123 & 34 & 43 & 44 & 45 & 92 & 100 & 51 & 52 & 54 & 57 & 116 & 120 & 981 \\ 
    \hline
        
    \end{tabular}
   
\end{table}

Portanto:

$$\bar{x}=\frac{981}{20} = 49,05$$

\end{document}