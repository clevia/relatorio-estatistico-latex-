
\documentclass{book}
\usepackage[brazil]{babel}
\usepackage[utf8]{inputenc}
\usepackage{indentfirst}
\usepackage{multicol}
\usepackage{xcolor}
%\usepackage{graphcx}
\begin{document}

\tableofcontents	
\newpage
\chapter{O espírito de Ouija}

%\label{O espirito de Ouija}	

\begin{multicols}{2}

 
 Shhhh, não chore criança. Eu sei que está me ouvindo, então não chore. Somos amigos, não somos? Por que está balançando a cabeça negativamente? Acaso insinua que não somos amigos? Pare de chorar. Foi você quem me chamou aqui. Não se lembra? Pois bem, eu lembro muito bem. Foi quando você e seus amigos estavam brincando daquele jogo. Como se chama mesmo? Ah, sim, Ouija. Sua amiga falou para vocês não brincarem. Mas você quis fazer isso, quis continuar. Me perguntei por que, já que você não tinha nenhum parente próximo morto com quem quisesse conversar. Mas agora eu sei, você queria saber sobre o seu futuro. Se seria rico, famoso, quando e como morreria. Essas perguntas não devem ser feitas, nunca. Até um bebê sabe que não se deve brincar disso. Não se deve fazer esse tipo de pergunta para os espíritos. Eles mentem, minha criança.
 
 Entendo que às vezes a curiosidade fala mais alto que a razão e vocês seguiram em frente.
 
 Vocês continuaram perguntando e não tinham nenhuma resposta em retorno até que eu resolvi aparecer. Fiz um show completo de luzes e respondi a todas as suas perguntas e mesmo assim vocês quiseram sair da brincadeira. Logo agora que eu estava gostando do jogo e decidi ficar. Não se pode fazer essa desfeita.
 
 Fiquei então ao seu lado, não sei por que, mas gostei de você desde o começo. Senti sua excitação e seu medo crescendo a cada pergunta feita, a cada descoberta maravilhosa e a cada vez que eu movia o ponteiro. Senti quando fiquei ao seu lado e você se arrepiou com a minha presença. Você gostou disso. Não me diga o contrário. O que? Vai continuar chorando? Você já foi mais divertido que isso. Deixe-me continuar.
 
 Fiquei tão feliz quando você olhou em minha direção e me viu ali. Você foi o único a me notar. Eu estava apenas esperando, aguardando o que aconteceria e então você quis estragar o jogo, parando por ali mesmo e se despedindo de mim. Eu não podia deixar aquela oportunidade passar e fiquei junto a você, passei para o seu plano, vim lhe fazer companhia em seu corpo. No começo você lutou, foi divertido ver você se debatendo e chorando como uma menininha enquanto eu assumia seu corpo, pegava discretamente a faca que estava na cozinha e ia em direção à sua amiga. Aquela chata que não queria brincar desde o começo. Eu ouvia você gritando, sentia você se debatendo enquanto tentava me fazer parar. Eu urrei de felicidade enquanto sua garganta era cortada e sentia aquele líquido quente escorrer pelas nossas mãos. O cheiro era maravilhoso.
 
 Todos estavam quietos enquanto nos observavam. Acho que estavam gostando do espetáculo.
 
 Após o transe passar, eles começaram a correr como baratas tontas, tentando abrir portas e janelas. Ficaram tão surpresos quando descobriram que trancamos tudo. Eu gargalhei, eu gritei de tanta felicidade enquanto você continuava chorando. Sempre tão entediante e monótono falando "por favor, não faça isso, meus amigos. Por favor" ah, por favor digo eu. Foi você quem começou com tudo isso e estou apenas terminando. Aonde eu parei? Lembrei, continuando.
 
 O próximo seria o gordinho com os óculos. Ele foi o mais fácil de pegar, não conseguia correr que logo cansava. Era divertido ver aquilo. Seus amigos são estranhos... Ele pediu tanto para que você parasse e nós apenas rimos de seus pedidos idiotas e pegamos aquela faca. Ele tentou se defender, lembra que ele até urinou nas calças? Aquilo me irritou um pouco, pois sujou seus tênis novos e o cheiro foi terrível, mas mesmo assim continuamos rindo. O encarei por mais um momento enquanto ele continuava implorando e acabei com seu sofrimento, passando a faca em seu pescoço. Pode me chamar de misericordioso, já que a morte de todos eles foi rápida.
 
 A proxima, ah como eu me diverti na próxima. Ela correu tanto, se escondeu e tentou ligar para a polícia. "Desculpe, estamos muito ocupados aqui". Lembro de ter falado isso enquanto arrancava os fios da parede e ia lentamente em sua direção. Ela chorou enquanto olhava o telefone inutilizado no chão. Correu para o banheiro e tentou se trancar. Ninguém se esconde assim de mim. Como ela não sabia disso? Bem, fiquei um momento brincando com ela do outro lado da porta enquanto a ouvia procurar algo para se defender. Como se houvesse algo, que boba, não é mesmo? Depois de um tempo, tudo lá dentro ficou quieto e nós entramos. Ela estava ao lado da pia com um bastão em mãos. Como ela arrumou aquilo? De repente ela estava correndo em nossa direção, tentando nos acertar inutilmente. A derrubei e ela ficou quieta, esperando nosso próximo passo. Era esperta, admito. Por que eles não desistem logo? Decidi fazer tudo de um modo diferente. Decidi fazer você vir à luz. Instantaneamente você voltou e começamos a chorar, foi maçante
 
 assistir tudo isso, mas fiquei quieto. Ouvi você repetindo várias e várias vezes que não foi você, que algo tinha tomado seu corpo e que estava fazendo isso contra sua vontade. Você acreditou que fui embora. Você a fez acreditar que eu fui embora. Ela acreditou. Bati palmas internamente enquanto vocês conversavam sobre um jeito de sair dali, de pedir ajuda. Foi então que eu apareci novamente e segurei seu lindo pescoço. Vi seu olhar assustado se transformando em horror. Surpresa. Sorri novamente, como era boa a sensação de ouvir seus pedidos para que parasse, seu ar se perdendo aos poucos, o som dos ossos se quebrando nas nossas mãos. Fiquei tão eufórico que quando acabou, nos sentimos vazios por um momento. Mas nós lembramos que ainda tinha mais uma pessoa. Onde ela estava?
 
 Fomos a sua procura, em cada quarto era uma nova decepção, mas continuamos procurando, era divertido procurar. Fomos assobiando uma música que você gostava de ouvir quando ainda era criança. Dava um pouco de sono, mas até que era legal.
 
 Ouvimos movimentos no quarto da frente e andamos até lá. Ela estava quieta perto da cama, com nossa faca nas mãos, nos esperando. O que ela pensava que estava fazendo? Facas podiam matar. Avisamos isso a ela e ela apenas nos encarou e continuou calada, esperando nossos próximos movimentos. Comecei a andar, nos aproximando lentamente dela, a deixando sem saída. Ela então investiu contra nós e acertou a faca em suas costelas. Sentimos a dor, logo após veio o sangue, mas não ligamos para isso. Tiramos a faca de suas mãos facilmente e a esfaqueamos no coração. Seu olhar logo foi perdendo a vida e então ela caiu na cama, inerte.
 
 Sorrimos e fomos ao hospital. Precisávamos de um tratamento para aquela ferida, eu poderia fazer parar de sangrar, mas precisava parecer humano para que acreditassem.
 
 Os policiais logo vieram e começamos com nossa encenação. Choramos tanto e falamos que uma de nossas amigas estava brincando com Ouija e começou a agir estranho. Ela assassinou todos os nossos amigos e eu tive que mata-la para sobreviver. Eles então foram para a casa encontraram nossa obra espalhada pelos quartos. Seus olhares de surpresa, acho que eles também gostaram do que viram.
 
 Então ficamos ali, naquela cama esperando pela lenta recuperação. Vendo nos noticiários o que sua amiga fez, ouvindo todos lamentando por nós, os únicos sobreviventes. Nos desejando melhoras e que conseguíssemos seguir em frente depois "desse terrível massacre".
 
 E agora, depois de tudo o que passamos juntos, te dou alguns minutos de lucidez e você me aparece com uma lâmina e simplesmente corta os pulsos? Que droga pensa que está fazendo? Pare de chorar agora, seu maldito. Sei que está me ouvindo muito bem enquanto perde os sentidos lentamente nessa droga de banheiro de hospital. Você não vai morrer agora.
 
\end{multicols}

\section{Acompanhantes}

\begin{multicols}{2}
	
	A porta deslizou. Esse maldito som. Todos sabiam o que acontecia após o som, era hora da chegada. Eles vinham, ligavam suas máquinas, faziam suas coisas inúteis e partiam. Até esse momento não podíamos fazer muita coisa, nossa presença não era desejada. Com isso, o tédio era dominante.
	
	\begin{flushleft}
		
		Pouco antes da partida, tínhamos um debate. Qual de nós iria acompanhar? Quem seria o acompanhado? Todos os dias, a mesma rotina. Como amávamos essa rotina. O prazer de acompanha-los era gigantesco, mas não era permitido que mais de um fosse acompanhado. Às vezes tínhamos êxito em burlar a regra, mas caso fossemos descobertos por nosso superior, éramos banidos. Ninguém queria sair dali, então o mais comum era seguir o procedimento. Um por vez.
		
	\end{flushleft}
	
	Quando o acompanhado era recém-chegado, o processo era ainda mais prazeroso. Toda aquela inocência, a euforia de estar ali, de ter conseguido chegar ali. Pode não parecer, mas as pessoas ansiavam por serem inúteis. A concorrência era gigantesca, era comum ver os candidatos esperando nervosamente, entrando na sala proibida e saindo um pouco mais relaxados, como se a parte mais difícil tivesse passado.
	
	\textcolor{green}{ Engraçado mencionar que havia uma sala proibida. Proibida logo para nós, quem teria esse poder, não é mesmo? Porém, segundo o superior não podíamos interferir naquela sala. Ela era parte essencial para que todo o projeto seguisse corretamente, para que não pudéssemos atrapalha ar a sequência natural da escolha dos acompanhados. Era decidido quem entrava e quem saía, tudo naquela sala gélida e sem graça.}
	
	Seria difícil escolher quem foi o meu melhor acompanhado se não fosse por ela, a acompanhada número 2506. Não era possível saber o nome de nenhum, apenas o código de registro. Era sua primeira vez, nunca havia visto tanta empolgação em um invólucro de carne. Meu trabalho foi difícil, tive que encontrar um novo meio para execução. Tive que plantar felicidade, os outros caçoaram de mim quando contei. Entretanto, eu sabia que o resultado seria positivo.
	
	\textcolor{red}{Todos nós acompanhávamos com um único objetivo, não só naquela localidade, mas em qualquer outra, o objetivo era o mesmo. Estávamos naquela há 22 anos e nenhum de nós havia sido capaz de conquistar nosso único propósito,} era divertido sem a menor dúvida, mas a cobrança do superior estava nos deixando apreensivos. Era comum que muitos nunca conseguissem alcançar nosso objetivo com todos os acompanhados, mas em muitas outras localidades a taxa de sucesso era de 42\%, enquanto a nossa era 0\%.
	
	\textcolor{violet}{ Não teríamos resultado algum se não fosse por ela, ela nos forçou a melhorar. Nos primeiros acompanhamentos, muitos reclamavam dos frutos que não estavam sendo colhidos por nós sobre ela. Até que chegou minha vez. Fiz minha presença se tornar quase imperceptível, como se nada a acompanhasse. Fiz que ela tomasse decisões pensando que a faria feliz. Aos poucos ela foi afastando aquele sentimento do coração dela. Aquele que nós abominamos. Cresci sobre ela com a ambição de fazer o melhor para si. Seu relacionamento desmoronou, para o bem dela. Seus pais se afastaram, para o bem dela. Sua arrogância sucumbiu sua humildade.} 
\end{multicols}

	\begin{flushleft} Eu era alvo de piadas, me tornei o acompanhante exclusivo dela, por pura chacota. Era falado que ela estava se tornando cada vez mais um de nós e que iria me engolir. Não passava uma única sombra de preocupação sobre minha cabeça. Eles estavam esquecendo que estávamos lidando com humanos, humanos são frágeis e sentimentais.
\end{flushleft}	

	\begin{flushright} Em menos de 3 meses, os resultados começaram a surgir. A saudade do seu verdadeiro amor, a percepção que sua família se fragmentara, a dificuldade financeira devido a separação dos pais, a arrogância refletindo em seu ambiente de trabalho.
\end{flushright}

\begin{center}
	content...	Não vou dizer que ela foi fraca, não. Ela teve fibra, ficou 26 meses aguentando o inferno em sua vida. Foi abraçada pela Depressão, ela é a pior de nós, sabe? Não estávamos obtendo sucesso em fazê-la permanecer, mas com o meu novo processo ela perdurou, até o alcance de nosso único objetivo. Adicionamos mais uma alma ao nosso superior.
\end{center} 
	``Ele era uma \copyright coleção maravilhosa de almas devastadas, que se perderam dentro de si mesmo. \textregistered Desorientações caudadas por nós. O problema dos humanos é que eles sempre atribuem a culpa a própria espécie. Nós estamos em todo lugar, levamos às vezes anos para que nosso objetivo seja alcançado, mas não falhamos. Uma hora ou outra, um ser ou outro, chegam até ele.''
\S
\dag	
\ddag

{\color[rgb]{0.44,0,0.11}Clevia Bento }

\end{document}